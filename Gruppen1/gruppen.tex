\documentclass[]{article}
\usepackage{lmodern}
\usepackage{amssymb,amsmath}
\usepackage{ifxetex,ifluatex}


\usepackage[utf8]{inputenc}
\usepackage[english,russian,ukrainian]{babel}

\usepackage{fixltx2e} % provides \textsubscript
\ifnum 0\ifxetex 1\fi\ifluatex 1\fi=0 % if pdftex
  \usepackage[T1]{fontenc}
  \usepackage[utf8]{inputenc}
\else % if luatex or xelatex
  \ifxetex
    \usepackage{mathspec}
  \else
    \usepackage{fontspec}
  \fi
  \defaultfontfeatures{Ligatures=TeX,Scale=MatchLowercase}
\fi
% use upquote if available, for straight quotes in verbatim environments
\IfFileExists{upquote.sty}{\usepackage{upquote}}{}
% use microtype if available
\IfFileExists{microtype.sty}{%
\usepackage{microtype}
\UseMicrotypeSet[protrusion]{basicmath} % disable protrusion for tt fonts
}{}
\usepackage[unicode=true]{hyperref}
\hypersetup{
            pdfborder={0 0 0},
            breaklinks=true}
\urlstyle{same}  % don't use monospace font for urls
\usepackage{graphicx,grffile}
\makeatletter
\def\maxwidth{\ifdim\Gin@nat@width>\linewidth\linewidth\else\Gin@nat@width\fi}
\def\maxheight{\ifdim\Gin@nat@height>\textheight\textheight\else\Gin@nat@height\fi}
\makeatother
% Scale images if necessary, so that they will not overflow the page
% margins by default, and it is still possible to overwrite the defaults
% using explicit options in \includegraphics[width, height, ...]{}
\setkeys{Gin}{width=\maxwidth,height=\maxheight,keepaspectratio}
\IfFileExists{parskip.sty}{%
\usepackage{parskip}
}{% else
\setlength{\parindent}{0pt}
\setlength{\parskip}{6pt plus 2pt minus 1pt}
}
\setlength{\emergencystretch}{3em}  % prevent overfull lines
\providecommand{\tightlist}{%
  \setlength{\itemsep}{0pt}\setlength{\parskip}{0pt}}
\setcounter{secnumdepth}{0}
% Redefines (sub)paragraphs to behave more like sections
\ifx\paragraph\undefined\else
\let\oldparagraph\paragraph
\renewcommand{\paragraph}[1]{\oldparagraph{#1}\mbox{}}
\fi
\ifx\subparagraph\undefined\else
\let\oldsubparagraph\subparagraph
\renewcommand{\subparagraph}[1]{\oldsubparagraph{#1}\mbox{}}
\fi

\date{}

\usepackage{multicol}

\usepackage{enumitem}
\makeatletter
\newcommand{\xslalph}[1]{\expandafter\@xslalph\csname c@#1\endcsname}
\newcommand{\@xslalph}[1]{%
    \ifcase#1\or а\or б\or в\or г\or д\or e\or є\or ж\or з\or i%
    \or й\or к\or л\or м\or н\or о\or п\or р\or с\or т%
    \or у\or ф\or х\or ц\or ч\or ш\or ю\or я\or аа\or бб\or вв %
    \else\@ctrerr\fi%
}
\AddEnumerateCounter{\xslalph}{\@xslalph}{m}
\makeatother


\begin{document}

Напишіть та протестуйте всі пункти даних задач. 
Результат розв'язків даних задач повинен міститись в даній папці за назвою "Hakaton1"
та запускатись даним мейкфайлом.
В заголовні файли можна додавати щось, але не можна видаляти чи змінювати.

Увага! Користуватись колекціями С++ не можна - задачі на "чистому" С!


В наступних двох задачах умови $Q(x)$ задаються предікатом, або 
булевою функцією, що передається через вказівник як аргумент.
\begin{enumerate}
\item 
Нехай множина дійсних чисел (float) задана у файлі. 
Дані вводяться у файл з клавіатури або з файлу input.dat, 
виводяться на екран та у файл output.txt.

Формат файлу:
На початку - unsigned число N - кількість чисел ($1 \le N \le 100000$). Далі
записано N float чисел, кожне не перевищує $2 * 10^9$ по модулю.

Сформуйте файл множини, тобто жоден елемент не повторюється.
Інтерфейс:

int formSetFile(const char* finput, const char* foutput)

Визначити:
\begin{enumerate}[label=\xslalph*)]
\item процедуру введення множини з консолі;
\item процедуру виведення множини на консоль;
\item процедуру доповнення множини;
\item процедуру видалення елемента з множини;
\item функцію, що дає відповідь, чи входить елемент до множини;
\item функцію, що дає відповідь, чи порожня множина;
\item функцію, що знаходить максимальний елемент множини;
\item функцію, що знаходить мінімальний елемент множини;
\item процедуру об'єднання множин;
\item процедуру різниці множин;
\item процедуру перетину множин;
\item функцію обчислення зваженої сумарної ваги множини (множина та масив дійсних елементів - ваг кожного елементу множини);
\item функцію обчислення діаметра множини;
\item функцію, що за множиною A знаходить підмножину всіх таких її
елементів, для яких справедлива умова $Q(х)$, $x\in A$;
\item функцію, що з'ясовує, чи є множина A підмножиною множини В;
\item функцію, що з'ясовує, чи дорівнює множина A множині В.
\end{enumerate}

\item
Дано бінарний файл, компоненти якого є записи (koef, st) -- дійсний коефіцієнт і
ступінь членів полінома ($koef \neq 0$, $koef \in \bf{R}$). Визначити підпрограми для виконання
таких дій над поліномом:
\begin{enumerate}[label=\xslalph*)]
\item введення полінома з консолі;
\item друк полінома (з файлу);
\item обчислення похідної від полінома (результат у файл з заданою назвою);
\item обчислення невизначеного інтеграла від полінома(результат у файл з заданою назвою, вільний член - 0);
\item упорядкування за степенями елементів полінома;
\item приведення подібних серед елементів полінома;
\item додавання, віднімання двох поліномів;
\item множення двох поліномів;
\item знаходження частки та залишку від ділення двох поліномів;
\item знаходження полінома за лінійної заміни змінної $x = dx + c$, $d \neq 0$;
\item знаходження полінома за заміни змінної $x = x^{d}$, $d \neq 0$;
\item знаходження ступеня поліному;
\item з'ясування, чи має поліном корені, рівні нулю, і визначення їхньої
кратності;
\item знаходження номеру максимального коефіцієнта серед
коефіцієнтів полінома, які задовольняють умову $Q(x)$ 
(якщо такого коефіцієнта немає - отримати -1);
\item знаходження номеру та значення мінімального за умовою $Q(х)$ коефіцієнта серед
коефіцієнтів полінома (якщо такого коефіцієнта немає - отримати
пару (-1,NULL) );
\item знаходження значення полінома в заданій точці.
\end{enumerate}

Додаткова інформація в заголовних файлах.

\end{enumerate}

\end{document}
